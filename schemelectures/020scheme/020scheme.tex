\documentclass{beamer}

\usepackage{amsmath}
\usepackage{alltt}
\usepackage{tikz}
\usepackage{tikz-qtree}
\usepackage{fancyvrb}

\usepackage{color}
\newcommand{\red}[1]{{\color{red}#1}}
\newcommand{\cyan}[1]{{\color{cyan}#1}}
\newcommand{\blue}[1]{{\color{blue}#1}}
\newcommand{\magenta}[1]{{\color{magenta}#1}}
\newcommand{\yellow}[1]{{\color{yellow}#1}}
\newcommand{\green}[1]{{\color{green}#1}}

\newcommand{\sect}[1]{
\section{#1}
\begin{frame}[fragile]\frametitle{#1}
}

\newcommand{\bi}{\begin{itemize}}
\newcommand{\ii}{\item}
\newcommand{\ei}{\end{itemize}}


\title
{
Scheme Notes 02
}

\subtitle{
} % (optional)

\author[Geoffrey Matthews]
{Geoffrey Matthews}
% - Use the \inst{?} command only if the authors have different
%   affiliation.

\institute[WWU/CS]
{
  Department of Computer Science\\
  Western Washington University
}
\begin{document}
\begin{frame}
\titlepage
\end{frame}

\sect{Procedures in Scheme}
\begin{alltt}
(define f
  \blue{(lambda (x) (* x x))}
  )


(\blue{f} 3) => 9
(\blue{f} 4) => 16
  
(\blue{(lambda (x) (* x x))} 3) => 9
(\blue{(lambda (x) (* x x))} 4) => 16

\end{alltt}
\end{frame}

\sect{Local Variables in Procedures: Closures}

\begin{alltt}
(define counter
  (let ((\red{n} 0))
    \blue{(lambda ()
      (set! \red{n} (+ \red{n} 1))
      \red{n}
    )}
  )
)

(counter)   =>  1
(counter)   =>  2
(counter)   =>  3
\end{alltt}
\end{frame}

\sect{A Procedure that Returns a Procedure}

\begin{alltt}
  (define curry
    (lambda (\red{a})
      \blue{(lambda (b)
        (+ \red{a} b)
      )}
    )
  )


  \blue{(curry 3)}       =>  #<procedure>

  (\blue{(curry 3)} 4)   =>  7

\end{alltt}
\end{frame}
\sect{A Procedure that Returns a Counter}

\begin{alltt}
(define new-counter
  (lambda ()
    (let ((\red{n} 0))
      \blue{(lambda ()
        (set! \red{n} (+ \red{n} 1))
        \red{n})}    )  )  )

(define x (new-counter))
(define y (new-counter))

(x)  => 1
(x)  => 2
(x)  => 3
(y)  => 1
(y)  => 2
(x)  => 4
\end{alltt}
\end{frame}

\sect{Procedures as Returned Values: Derivatives}
\begin{alltt}
(define d
  (lambda (f)
    (let* ((delta 0.00001)
           (two-delta (* 2 delta)))
      \blue{(lambda (x)
        (/ (- (f (+ x delta)) (f (- x delta)))
           two-delta))})))



((d (lambda (x) (* x x x)))  5)    => ?
((d sin) 0.0)                      => ?
           
\end{alltt}
\end{frame}

\sect{Procedures as Returned Values: Procedures as Data}
\begin{alltt}

(define make-pair
  (lambda (a b)
    \blue{(lambda (i)
      (cond ((= i 0) a)
            ((= i 1) b)
            (else (error 'bad-index))))}
  ))

(define x (make-pair 4 8))
(define y (make-pair 100 200))
(define z (make-pair x y))  

(x 0)       => ?
(y 1)       => ?
((z 1) 0)   => ?
\end{alltt}
\end{frame}


\sect{Procedures as parameters}

Summation notation: \[\sum_{i=a}^{b} f(i) = f(a) + \ldots + f(b)\]

In scheme:
\begin{alltt}
(define sum
  (lambda (a b \blue{f})
    (if (> a b)
        0
        (+ (\blue{f} a) (sum (+ a 1) b \blue{f})))))

(sum 1 10 \blue{square}) => 385
(sum 1 10 \blue{(lambda (x) (* x x x))}) => 3025
\end{alltt}

\end{frame}

\sect{Better notation for summations}

Instead of
\[\sum_{i=a}^{b} f(i) = f(a) + \ldots + f(b)\]
use
\[\sum_{a}^{b} f = f(a) + \ldots + f(b)\]
Why don't we use that?
\pause

Because then you have problems with
\[\sum_{i=a}^{b} i^2 = a^2 + \ldots + b^2\]
\[\sum_{a}^{b} \fbox{?} = a^2 + \ldots + b^2\]
\pause
\[\sum_{a}^{b} \lambda i . i^2 = a^2 + \ldots + b^2\]

\end{frame}

\sect{Matches Scheme Code Better, Too}


\[\sum_{a}^{b} f = f(a) + \ldots + f(b)\]

\begin{alltt}
(define sum
  (lambda (a b \blue{f})
    (if (> a b)
        0
        (+ (\blue{f} a) (sum (+ a 1) b \blue{f})))))
\end{alltt}

\end{frame}

\sect{We have the same problem with derivatives}

\begin{align*}
  \frac{d}{dx} f(x) &= f'(x)\\\\
  \frac{d x^2}{dx}  &= 2x \\\\\\
D(f) &= f'\\\\
D( \lambda x . x^2 )&= \lambda x . 2x
\end{align*}

\end{frame}

\sect{The chain rule}
With pure functions it's easy:
\begin{align*}
  D  (f\circ g) = (D (f) \circ g) \cdot D (g)
\end{align*}
With applied functions:
\begin{align*}
  F &= f\circ g\\
  F(x) &= f(g(x))\\
  F'(x) &= f'(g(x))g'(x)\\
  \frac{d F(x)}{dx} &= \frac{d f(g(x))}{d g(x)}\cdot \frac{d g(x)}{dx}
\end{align*}
Or, if we let $z=f(y)$ and $y=g(x)$ (which is weird) then it looks
kind of nice and is easy to memorize (cancel fractions!):
\begin{align*}
  \frac{dz}{dx} &= \frac{dz}{dy} \cdot \frac{dy}{dx}
\end{align*}


\end{frame}


\sect{Finding fixed points}

$x$ is a {\sl fixed point} of $f$ if $x = f(x)$

For some functions you can find fixed points by iterating:
\\
$x, f(x), f(f(x)), f(f(f(x))), \ldots$

\end{frame}
\sect{Fixed points in scheme:}
{\scriptsize
\begin{alltt}
(define fixed-point
  (lambda (f)
    (let 
        ((tolerance 0.0001)
         (max-iterations 10000))
      (letrec 
          ((close-enough? 
            (lambda (a b) (< (abs (- a b)) tolerance)))
           (try 
            (lambda (guess iterations)
              (let ((next (f guess)))
                (cond ((close-enough? guess next) next)
                      ((> iterations max-iterations) #f)
                      (else (try next (+ iterations 1)))))))
           )
        (try 1.0 0)))))

(fixed-point cos)  => 0.7390547907469174
(fixed-point sin)  => 0.08420937654137994
(fixed-point (lambda (x) x)) => 1.0
(fixed-point (lambda (x) (+ x 1))) => #f
\end{alltt}
}

\end{frame}
\sect{Remember Newton's Method?}
{\small
Newton's method: \\
If $y$ is a guess for $\sqrt{x}$, then 
the average of $y$ and $x/y$ is an even better guess.

\begin{tabular}{cccc}
$x$ & guess & quotient & average\\
2 & 1.0 & 2.0 & 1.5\\
2 & 1.5 & 1.3333333333333333 & 1.4166666666666665\\
2 & 1.4166666666666665 & 1.411764705882353 & 1.4142156862745097\\
2 & 1.4142156862745097 & 1.41421143847487 & 1.4142135623746899\\
...
\end{tabular}
Evidently, we want to iterate, and keep recomputing
these things until we find a value that's close enough.
}

\end{frame}
\sect{Newton's Method Using Functional Programming}

\begin{alltt}
(define sqrt
  (lambda (x)
    (fixed-point
      (lambda (y)
        (/ 
           (+ y (/ x y)) 
           2
        )))))
\end{alltt}

\end{frame}



\sect{Procedures as Returned Values: Newton's Method Again}

\begin{alltt}
(define average-damp 
  (lambda (f)
    (lambda (x) (/ (+ x (f x)) 2))))

(define sqrt
  (lambda (x)
    (fixed-point (average-damp (lambda (y) (/ x y))))))
\end{alltt}


\end{frame}
\end{document}
