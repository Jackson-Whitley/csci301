\documentclass{beamer}

\usepackage{amsmath}
\usepackage{alltt}
\usepackage{tikz}
\usepackage{tikz-qtree}
\usepackage{fancyvrb}

\usepackage{color}
\newcommand{\red}[1]{{\color{red}#1}}
\newcommand{\cyan}[1]{{\color{cyan}#1}}
\newcommand{\blue}[1]{{\color{blue}#1}}
\newcommand{\magenta}[1]{{\color{magenta}#1}}
\newcommand{\yellow}[1]{{\color{yellow}#1}}
\newcommand{\green}[1]{{\color{green}#1}}

\newcommand{\sect}[1]{
\section{#1}
\begin{frame}[fragile]\frametitle{#1}
}

\newcommand{\bi}{\begin{itemize}}
\newcommand{\ii}{\item}
\newcommand{\ei}{\end{itemize}}


\title
{
Scheme Notes 01
}

\subtitle{
} % (optional)

\author[Geoffrey Matthews]
{Geoffrey Matthews}
% - Use the \inst{?} command only if the authors have different
%   affiliation.

\institute[WWU/CS]
{
  Department of Computer Science\\
  Western Washington University
}
\begin{document}
\begin{frame}
\titlepage
\end{frame}



\sect{Resources}
\bi
\ii The software:
\bi\ii \url{https://racket-lang.org/}\ei
\ii Texts:
\bi
\ii \url{https://mitpress.mit.edu/sicp/}
\ii \url{http://www.scheme.com/tspl3/}\\ (make sure you use the 3rd
edition and not the 4th)
\ii \url{http://ds26gte.github.io/tyscheme/}
\ei
\ei
\end{frame}


\sect{Running the textbook examples}

\bi
\ii Using the racket language is usually best, the examples from
    {\em The Scheme Programming Language} should run without modification.

 \ii The examples from SICP are a little more idiosyncratic.  Most of
 them can be run by installing the {\tt sicp} package as in
 these instructions:\\
 \url{http://stackoverflow.com/questions/19546115/which-lang-packet-is-proper-for-sicp-in-dr-racket}

 \ei
\end{frame}

\sect{Simple Scheme Program}
\VerbatimInput[frame=single,firstline=15,label=quadratic.rkt]{quadratic.rkt}
\end{frame}

\sect{Simple Scheme Program Unit Tests}
\VerbatimInput[frame=single,firstline=5,label=quadratic-test.rkt]{quadratic-test.rkt}
\end{frame}




\sect{There are two types of expressions:}

\bi
\ii
   {\bf Primitive expressions:}  no parentheses.

\blue{Constants:}

 \verb|4   3.141592   #t   #f    "Hello world!"|

\blue{Variables:}

 \verb|x    long-variable-name      a21|

 \vfill

\ii {\bf Compound expressions:}  with parentheses.

  \blue{Special forms:}

  \verb|(define x 99)|

  \verb|(if (> x y) x y)|
  
\blue{Function calls:}

\verb|(+ 1 2 3)|

\verb|(list (list 1 2) (list 3 4))|
\ei

\end{frame}

\sect{Introducing Local Variables}

\begin{alltt}
(let ((x 3)
      (y 4)
      (z 5))
    (+ x (* y z)))   =>  23
\end{alltt}


\end{frame}
\sect{Beware!  This will NOT work.}

\begin{alltt}
(let ((x 3)
      (y (* 2 x))
      (z (* 3 x)))
    (+ x (* y z)))   =>  57
\end{alltt}


\end{frame}
\sect{But this will.}

\begin{alltt}
(\red{let*} ((x 3)
       (y (* 2 x))
       (z (* 3 x)))
    (+ x (* y z)))   =>  57
\end{alltt}


\end{frame}
\sect{Defining Procedures}

Two equivalent ways:
\begin{alltt}
(define \red{(square x)} (* x x))
(define square (lambda (x) (* x x)))
\end{alltt}
The first one is more in line with the procedure call:
\begin{alltt}
\red{(square 5)} => 25
\end{alltt}

\end{frame}
\sect{Defining Procedures}

Two equivalent ways:
\begin{alltt}
(define (square x) (* x x))
(define square \red{(lambda (x) (* x x))})
\end{alltt}
The second one is more in line with defining other things:
\begin{alltt}
(define x \red{(* 3 4)})
(define y \red{(list 5 9 22)})
\end{alltt}
The action of {\tt define} is simply to give a {\sl name} to
the result of an expression.

\end{frame}
\sect{Defining Procedures}

Two equivalent ways:
\begin{alltt}
(define (square x) (* x x))
(define square \red{(lambda (x) (* x x))})
\end{alltt}
The result of a lambda-expression is an anonymous function.
\\
We can name it, as above, or use it without any name at all:
\begin{alltt}
(square 5) => 25
(\red{(lambda (x) (* x x))} 5) => 25
\end{alltt}

\end{frame}
\sect{Procedures always return a value}
\begin{alltt}
(define (bigger a b c d)
  (if (> a b) \red{c d}))  



(define (solve-quadratic-equation a b c)
  (let ((disc (sqrt (- (* b b)
                        (* 4.0 a c)))))
    \red{(list     
     (/ (+ b disc)
        (* 2.0 a))
     (/ (+ (- b) disc)
        (* 2.0 a)))}
    ))
\end{alltt}
\end{frame}
\sect{Solving problems}
{\small
Newton's method: \\
If $y$ is a guess for $\sqrt{x}$, then 
the average of $y$ and $x/y$ is an even better guess.

\begin{tabular}{cccc}
$x$ & guess & quotient & average\\
2 & 1.0 & 2.0 & 1.5\\
2 & 1.5 & 1.3333333333333333 & 1.4166666666666665\\
2 & 1.4166666666666665 & 1.411764705882353 & 1.4142156862745097\\
2 & 1.4142156862745097 & 1.41421143847487 & 1.4142135623746899\\
...
\end{tabular}
Evidently, we want to iterate, and keep recomputing
these things until we find a value that's close enough.
}

\end{frame}

\sect{Newton's Method in Scheme}
{\scriptsize
\begin{verbatim}
(define sqrt-iter 
  (lambda (guess x)
    (if (good-enough? guess x)
        guess
        (sqrt-iter (improve guess x) x))))

(define improve 
  (lambda (guess x)
    (average guess (/ x guess))))

(define average 
  (lambda (x y) (/ (+ x y) 2)))

(define good-enough? 
  (lambda (guess x)
    (< (abs (- (square guess) x)) 0.00001)))

(define square 
  (lambda (x) (* x x)))

(define sqrt 
  (lambda (x) (sqrt-iter 1.0 x)))
\end{verbatim}
}
Decompose big problems into smaller problems.

\end{frame}

\sect{Newton's Method in Scheme}
{\scriptsize
\begin{verbatim}
(define sqrt-iter 
  (lambda (guess x)
    (if (good-enough? guess x)
        guess
        (sqrt-iter (improve guess x) x))))

(define improve 
  (lambda (guess x)
    (average guess (/ x guess))))

(define average 
  (lambda (x y) (/ (+ x y) 2)))

(define good-enough? 
  (lambda (guess x)
    (< (abs (- (square guess) x)) 0.00001)))

(define square 
  (lambda (x) (* x x)))

(define sqrt 
  (lambda (x) (sqrt-iter 1.0 x)))
\end{verbatim}
}
Note:  NO GLOBAL VARIABLES!

\end{frame}
\sect{Definitions can be nested}
\begin{alltt}
(define sqrt 
  (lambda (x)
\blue{    (define good-enough? 
      (lambda (guess x)
        (< (abs (- (square guess) x)) 0.001)))
    (define improve 
      (lambda (guess x)
        (average guess (/ x guess))))
    (define sqrt-iter 
      (lambda (guess x)
        (if (good-enough? guess x)
            guess
            (sqrt-iter (improve guess x) x))))}
    (sqrt-iter 1.0 x)))
\end{alltt}

\end{frame}
\sect{Parameters need not be repeated}
\begin{alltt}
(define sqrt 
  (lambda (\blue{x})
    (define good-enough? 
      (lambda (\red{guess})
        (< (abs (- (square \red{guess}) \blue{x})) 0.001)))
    (define improve 
      (lambda (\red{guess})
        (average \red{guess} (/ \blue{x} \red{guess}))))
    (define sqrt-iter 
      (lambda (\red{guess})
        (if (good-enough? \red{guess})
            \red{guess}
            (sqrt-iter (improve \red{guess})))))
    (sqrt-iter 1.0)))
\end{alltt}
\end{frame}
\sect{Introducing local functions with {\tt letrec} }
\begin{alltt}
(define sqrt 
  (lambda (x)
    (letrec ((good-enough? 
              (lambda (guess)
                (< (abs (- (square guess) x)) 0.001)))
             (improve 
              (lambda (guess)
                (average guess (/ x guess))))
             (sqrt-iter 
              (lambda (guess)
                (if (good-enough? guess)
                    guess
                    (sqrt-iter (improve guess)))))
             )
             (sqrt-iter 1.0))))
\end{alltt}

\end{frame}


\sect{Procedures in Scheme}
\begin{alltt}
(define f
  \blue{(lambda (x) (* x x))}
  )


(\blue{f} 3) => 9
(\blue{f} 4) => 16
  
(\blue{(lambda (x) (* x x))} 3) => 9
(\blue{(lambda (x) (* x x))} 4) => 16

\end{alltt}
\end{frame}

\sect{Local Variables in Procedures: Closures}

\begin{alltt}
(define counter
  (let ((\red{n} 0))
    \blue{(lambda ()
      (set! \red{n} (+ \red{n} 1))
      \red{n}
    )}
  )
)

(counter)   =>  1
(counter)   =>  2
(counter)   =>  3
\end{alltt}
\end{frame}

\sect{A Procedure that Returns a Procedure}

\begin{alltt}
  (define curry
    (lambda (\red{a})
      \blue{(lambda (b)
        (+ \red{a} b)
      )}
    )
  )


  \blue{(curry 3)}       =>  #<procedure>

  (\blue{(curry 3)} 4)   =>  7

\end{alltt}
\end{frame}
\sect{A Procedure that Returns a Counter}

\begin{alltt}
(define new-counter
  (lambda ()
    (let ((\red{n} 0))
      \blue{(lambda ()
        (set! \red{n} (+ \red{n} 1))
        \red{n})}    )  )  )

(define x (new-counter))
(define y (new-counter))

(x)  => 1
(x)  => 2
(x)  => 3
(y)  => 1
(y)  => 2
(x)  => 4
\end{alltt}
\end{frame}

\sect{Procedures as Returned Values: Derivatives}
\begin{alltt}
(define d
  (lambda (f)
    (let* ((delta 0.00001)
           (two-delta (* 2 delta)))
      \blue{(lambda (x)
        (/ (- (f (+ x delta)) (f (- x delta)))
           two-delta))})))



((d (lambda (x) (* x x x)))  5)    => ?
((d sin) 0.0)                      => ?
           
\end{alltt}
\end{frame}

\sect{Procedures as Returned Values: Procedures as Data}
\begin{alltt}

(define make-pair
  (lambda (a b)
    \blue{(lambda (i)
      (cond ((= i 0) a)
            ((= i 1) b)
            (else (error 'bad-index))))}
  ))

(define x (make-pair 4 8))
(define y (make-pair 100 200))
(define z (make-pair x y))  

(x 0)       => ?
(y 1)       => ?
((z 1) 0)   => ?
\end{alltt}
\end{frame}


\sect{Procedures as parameters}

Summation notation: \[\sum_{i=a}^{b} f(i) = f(a) + \ldots + f(b)\]

In scheme:
\begin{alltt}
(define sum
  (lambda (a b \blue{f})
    (if (> a b)
        0
        (+ (\blue{f} a) (sum (+ a 1) b \blue{f})))))

(sum 1 10 \blue{square}) => 385
(sum 1 10 \blue{(lambda (x) (* x x x))}) => 3025
\end{alltt}

\end{frame}

\sect{Better notation for summations}

Instead of
\[\sum_{i=a}^{b} f(i) = f(a) + \ldots + f(b)\]
use
\[\sum_{a}^{b} f = f(a) + \ldots + f(b)\]
Why don't we use that?
\pause

Because then you have problems with
\[\sum_{i=a}^{b} i^2 = a^2 + \ldots + b^2\]
\[\sum_{a}^{b} \fbox{?} = a^2 + \ldots + b^2\]
\pause
\[\sum_{a}^{b} \lambda i . i^2 = a^2 + \ldots + b^2\]

\end{frame}

\sect{Matches Scheme Code Better, Too}


\[\sum_{a}^{b} f = f(a) + \ldots + f(b)\]

\begin{alltt}
(define sum
  (lambda (a b \blue{f})
    (if (> a b)
        0
        (+ (\blue{f} a) (sum (+ a 1) b \blue{f})))))
\end{alltt}

\end{frame}

\sect{Finding fixed points}

$x$ is a {\sl fixed point} of $f$ if $x = f(x)$

For some functions you can find fixed points by iterating:
\\
$x, f(x), f(f(x)), f(f(f(x))), \ldots$

\end{frame}
\sect{Fixed points in scheme:}
{\scriptsize
\begin{alltt}
(define fixed-point
  (lambda (f)
    (let 
        ((tolerance 0.0001)
         (max-iterations 10000))
      (letrec 
          ((close-enough? 
            (lambda (a b) (< (abs (- a b)) tolerance)))
           (try 
            (lambda (guess iterations)
              (let ((next (f guess)))
                (cond ((close-enough? guess next) next)
                      ((> iterations max-iterations) #f)
                      (else (try next (+ iterations 1)))))))
           )
        (try 1.0 0)))))

(fixed-point cos)  => 0.7390547907469174
(fixed-point sin)  => 0.08420937654137994
(fixed-point (lambda (x) x)) => 1.0
(fixed-point (lambda (x) (+ x 1))) => #f
\end{alltt}
}

\end{frame}
\sect{Remember Newton's Method?}
{\small
Newton's method: \\
If $y$ is a guess for $\sqrt{x}$, then 
the average of $y$ and $x/y$ is an even better guess.

\begin{tabular}{cccc}
$x$ & guess & quotient & average\\
2 & 1.0 & 2.0 & 1.5\\
2 & 1.5 & 1.3333333333333333 & 1.4166666666666665\\
2 & 1.4166666666666665 & 1.411764705882353 & 1.4142156862745097\\
2 & 1.4142156862745097 & 1.41421143847487 & 1.4142135623746899\\
...
\end{tabular}
Evidently, we want to iterate, and keep recomputing
these things until we find a value that's close enough.
}

\end{frame}
\sect{Newton's Method Using Functional Programming}

\begin{alltt}
(define sqrt
  (lambda (x)
    (fixed-point
      (lambda (y)
        (/ 
           (+ y (/ x y)) 
           2
        )))))
\end{alltt}

\end{frame}

\sect{General instructions for functional programming}
\centerline{\fbox{\fbox{\Large \bf Use lots of functions!}}}
\vfill


Unless specified otherwise:
\begin{itemize}
\item {\bf NO} assignment statements.
  \begin{itemize}
  \item Use {\tt let} or parameters in function calls
  \item Exception: implementing objects with closures.
  \end{itemize}
\item {\bf NO} {\tt for} loops.
  \begin{itemize}
  \item Use function calls.
  \end{itemize}
\item {\bf NO} {\tt while} loops.
  \begin{itemize}
  \item Use function calls.
  \end{itemize}
\item {\bf NO} nested {\tt if} statements.
  \begin{itemize}
  \item Use {\tt cond}
  \end{itemize}
\end{itemize}


\end{frame}






\end{document}
