\documentclass{article}
\pagestyle{empty}
\usepackage[margin=1in]{geometry}
\usepackage{fancyvrb,amsmath}
\usepackage{multicol}
\usepackage{xcolor}

\newcommand{\set}[1]{\ensuremath{\{#1\}}}
\newcommand{\power}[1]{\ensuremath{\mathcal{P}(#1)}}

\title{CSCI 301, Lab \# 8}
\author{Spring 2018}
\date{}
\begin{document}

\maketitle
%\setlength{\columnsep}{2em}
%\begin{multicols}{2}


\paragraph{Goal:} This is the last in a series of five labs that will
build an interpreter for Scheme.  In this lab we will add the {\tt letrec}
special form.

\paragraph{Due:} Your program, named {\tt lab08.rkt}, must be submitted to
Canvas before midnight, Monday, June 4.

\paragraph{Unit tests:}
At a minimum, your program must pass the unit tests found in the
file {\tt lab08-test.rkt}.  Place this file in the same folder
as your program, and run it;  there should be no output.  Include
your unit tests in your submission.

\paragraph{Letrec creates closures that include their own definitions.}

Consider a typical application of {\tt letrec}:
\begin{Verbatim}[frame=single]
  (letrec ((plus (lambda (a b) (if (= a 0) b (+ 1 (plus (- a 1) b)))))
           (even? (lambda (n) (if (= n 0) true (odd? (- n 1)))))
           (odd? (lambda (n) (if (= n 0) false (even? (- n 1))))))
    (even? (plus 4 5)))
\end{Verbatim}
{\tt plus} is a straightforward recursive function.  {\tt even?} and
{\tt odd?} are mutually recursive functions, each one requires the
other.

If we evaluate the {\tt lambda} forms in  the current environment,
none of the three functions will be defined in that environment (or
else they will have the {\em wrong} definitions).

We want the closures to close over an environment in which {\tt plus},
{\tt even?} and {\tt odd?} are defined.  To do this, we will follow
this strategy.
\begin{enumerate}
  \item When we evaluate the {\tt letrec} special form, we do so in
    the context of some environment.  Let's call this the {\tt
      OldEnv}.
  \item We now create a dummy environment (which can be an empty
    list), to use in the next step.  Let's call this {\tt DummyEnv}.
  \item We now create closures in the {\tt letrec} form
    using {\tt DummyEnv}.  In other words,
    the {\em saved} environment in each closure is {\tt DummyEnv}.
  \item Now create a {\em new} environment where the symbols
    in the {\tt letrec} form (in this case,
    {\tt plus}, {\tt even?} and {\tt odd?}) are bound to these closures.

    Do not add anything to this environment except what the
    variable-value combinations from the {\tt letrec} form!

    Let's call this environment the {\tt LetrecEnv}.

  \item We now have three environments we're dealing with:
    {\tt OldEnv}, {\tt DummyEnv}, and {\tt LetrecEnv}.  Make sure you
    know what each of these is.

  \item Create a fourth environment by appending {\tt LetrecEnv} to
    the front of {\tt OldEnv}.  Let's call this one {\tt NewEnv}.

\item Now go through the closures  in {\tt LetrecEnv}
  and {\em change} the {\em saved} environment in each closure
  to {\tt NewEnv}.  Make sure you only go through the newly created
  closures in {\tt LetrecEnv}, and not through all the closures 
  in {\tt OldEnv} or {\tt NewEnv}.  (Note that {\tt NewEnv} includes
  {\tt OldEnv}, but {\tt LetrecEnv} does not.)

Consult the {\bf Racket} documentation to see how to
change a field of a structure.  You will have to change the
structure definition to make a mutable field:
\begin{Verbatim}[frame=single]
  (struct closure (vars body (env #:mutable)))
\end{Verbatim}

\item We can now evaluate the body of the {\tt letrec} form, using
  {\tt NewEnv}!
  
\end{enumerate}


\paragraph{Some tricky examples.}

\begin{Verbatim}[frame=single,numbers=left]
  (letrec ((f
              (lambda (x) (if (= 0 x) 1 (* x (f (- x 1))))))) ; f from line 1
      (f                                                      ; f from line 1
         (let ((f
                  (lambda (x) (* 3 x))))
             (let ((f
                      (lambda (x) (f (f x)))))                ; f from line 4
                 (f 2)))))                                    ; f from line 6
\end{Verbatim}

\begin{Verbatim}[frame=single,numbers=left]
  (letrec ((f
              (lambda (x) (if (= 0 x) 1 (+ x (f (- x 1))))))) ; f from line 1
      (f                                                      ; f from line 1
         (let ((f 
                  (lambda (x) (f (+ x 2)))))                  ; f from line 1
            (f                                                ; f from line 4
               (let ((f 
                        (lambda (x) (f (+ x 3)))))            ; f from line 4
                  (f 3))))))                                  ; f from line 6
\end{Verbatim}

\end{document}

