\documentclass{article}
\pagestyle{empty}
\usepackage[margin=1in]{geometry}
\usepackage{fancyvrb,amsmath}
\usepackage{multicol}

\newcommand{\set}[1]{\ensuremath{\{#1\}}}
\newcommand{\power}[1]{\ensuremath{\mathcal{P}(#1)}}

\title{CSCI 301, Lab \# 6}
\author{Spring 2018}
\date{}
\begin{document}

\maketitle
%\setlength{\columnsep}{2em}
%\begin{multicols}{2}


\paragraph{Goal:} This is the third in a series of five labs that will
build an interpreter for Scheme.  In this lab we will add the {\tt let}
special form.

\paragraph{Due:} Your program, named {\tt lab06.rkt}, must be submitted to
Canvas before midnight, Monday, May 21.

\paragraph{Unit tests:}
At a minimum, your program must pass the unit tests found in the
file {\tt lab06-test.rkt}.  Place this file in the same folder
as your program, and run it;  there should be no output.  Include
your unit tests in your submission.

\paragraph{Let:}

The special form {\tt let} has the following syntax, with a typical
example shown at right.  It consists of a list beginning with the
symbol {\tt let}, then a list of symbol-value lists, and finally a
single expression.  You will expand your {\tt evaluate-special-form}
function to handle this case, and also add code to the {\tt
  special-form?} boolean to recognize a {\tt let} form.  No other
changes need be made to your interpreter.
\begin{multicols}{2}
\begin{Verbatim}[frame=single]
  (let ((sym1  exp1)
        (sym2  exp2)
        ...         )
     expr)
\end{Verbatim}
\begin{Verbatim}[frame=single]
  (let ((x  (+ 2 2))
        (y  x)
        (z  (* 3 3)))
     (+ a x y z))
\end{Verbatim}
\end{multicols}
To evaluate this form, in an environment {\tt e1}, we evaluate
all the forms, {\tt exp1}, {\tt exp2}, ... in the environment {\tt
  e1}.   Note that this means {\tt y} will get the value {\tt x} had
in {\tt e1}, not 4.  Let's say that in {\tt e1} {\tt x} had the value
10 and {\tt a} had the value 20.

We now have a list of variables, {\tt (x y z)}, and a list of values,
{\tt (4 10 9)}.  We make a new environment by appending these
variables and their values to the {\em front} of {\tt e1}.

The single expression at the end of a {\t let} form is now evaluated
in this {\em new, extended} environment.  Thus, the final value will
be  {\tt (+ a x y z) => (+ 20 4 10 9) => 43}.  Make sure you
understand this example before proceeding.

It is also important to understand that this new, extended
environment, is {\em only } used to evaluate the {\tt expr} embedded
in the {\tt let } form.  After the {\tt let} form is evaluated, you go
back to using the old environment.  For example:
\begin{Verbatim}[frame=single]
  (let ((x 10)) (+ (let ((x 20)) (+ x x)) x))  =>  50
\end{Verbatim}
    In this
example, we bind {\tt x} to 10, then create a local environment in
which {\tt x } is bound to 20, in this new environment we evaluate
{\tt (+ x x)} to get 40, and then, outside of the new environment we
evaluate {\tt x} again, getting 10, which is added to 40 to get 50.

A slightly trickier example is this:
\begin{Verbatim}[frame=single]
  (let ((x 10)) (+ (let ((x (+ x x))) (+ x x)) x))
\end{Verbatim}
  What do you
think this will evaluate to?  Enter it into the Racket interpreter to
see if you really understand.  Your interpreter should get the same
result. 

In your implementation, all this tricky scoping will be handled by the
fact that the extended environment in the {\tt let} form is {\em only}
used to evaluate the included {\tt expr}.  It should not be visible
outside of handling the {\tt let} form.




\end{document}

\end
