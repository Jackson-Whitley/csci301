\documentclass{article}
\usepackage[margin=0.75in]{geometry}
\usepackage{fancyvrb}
\usepackage{multicol}
\title{CSCI 301, Lab \# 1}
\author{Spring 2018}
\date{}
\begin{document}

\maketitle
%\setlength{\columnsep}{2em}
\begin{multicols}{2}

\begin{description}
\item[Goal:] The purpose of this lab is write some simple code in {\em
  Scheme}, and to get familiar with the hardware and software that we
  will be using all quarter, the lab room, your TA, and the
  sumbmission procedure for Canvas.

\item[Due:] Your program, named {\tt lab01.rkt}, must be submitted to
  Canvas before midnight, Monday, Oct 1.

  \item[Unit tests:]
  At a minimum, your program must pass the unit tests found in the
  file {\tt lab01-test.rkt}.  Place this file in the same folder
  as your program, and run it;  there should be no output.

\item[Program:] Write a {\sc Scheme} procedure called {\tt make-pi} 
to compute $\pi$
  using the (slowly converging) series:
\[
{\pi} = 4 - \frac43 + \frac45 - \frac47 + \frac49 - \ldots
\]
The procedure takes one parameter, which will be the 
accuracy we need.  We can stop whenever the next factor we would
add is smaller than this accuracy.  For example:
\begin{Verbatim}[frame=single]
(make-pi 0.1) =>  3.09162380666784
(make-pi 0.001) =>  3.1410926536210413
(make-pi 0.0000001) => 3.141592603589817
\end{Verbatim}
Be careful!  This series converges {\em very} slowly.  If you're
curious, instrument the procedure so that it also prints out the
number of iterations it took to get the accuracy (this is optional).

Make sure that your program {\em returns} the value of $\pi$ computed,
and doesn't just print it.  For example, this should not give an
error: \verb|(+ (make-pi 0.1) (make-pi 0.1))|


To get the loop done, define a recursive procedure with (at least)
three
parameters that behave like this (I've truncated the decimals):\\
\begin{tabular}{rrr}
 Numerator & Denominator & Sum \\\hline
 4.0 & 1.0 & 0.0\\
 -4.0 & 3.0 & 4.0\\
 4.0 & 5.0 & 2.666\\
 -4.0 & 7.0 & 3.466\\
 4.0 & 9.0 & 2.895\\
 -4.0 & 11.0 & 3.339\\
 4.0 & 13.0 & 2.976\\
 -4.0 & 15.0 & 3.283\\
 4.0 & 17.0 & 3.017\\
 -4.0 & 19.0 & 3.252\\
 4.0 & 21.0 & 3.041\\
\end{tabular}

Make sure your program starts its loop with floating point numbers,
{\em e.g.} 4.0, 1.0, {\em etc.}  If you start with exact integers,
Scheme will try to keep exact rational numbers through all of those
computations and it will be substantially slower.  Also, your answers
will look like this:
\begin{Verbatim}[frame=single]
(make-pi 0.1) => 
    516197940314096/166966608033225
\end{Verbatim}

Name your program {\tt lab01.rkt} and include a comment block like the
one shown (with your own name and W number).  Also, hopefully, with
the correct code, although this one will pass the unit tests!

\VerbatimInput[frame=single,label=lab01.rkt]{comment-demo.rkt}

\end{description}
\end{multicols}

\end{document}

\end
